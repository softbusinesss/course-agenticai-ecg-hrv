\documentclass[aspectratio=169]{beamer}

% As per README instructions for the Nordling Lab template
\usetheme{NordlingLab169}

\usepackage[utf8]{inputenc}
\usepackage{graphicx}
\usepackage{booktabs}

% Title Slide Information
\title{HRV Coach Pro v2.1}
\subtitle{Autonomous Clinical Intelligence}
\author{Khan Saquib , Liu TzuEn Andrew , Peng ToChen }
\institute{ National Cheng Kung University}
\date{\today}

\begin{document}

% Slide 1: Title Page
\begin{frame}
    \titlepage

\end{frame}

% Slide 2: Project Overview
\begin{frame}{Project Overview}
    \begin{itemize}
        \item \textbf{HRV Coach Pro v2.1}: An autonomous agentic system for clinical HRV analysis.
        \item \textbf{Core Function}: Converts raw noisy ECG $\rightarrow$ validated HRV metrics $\rightarrow$ clinical-style insights.
        \item \textbf{Agentic Approach}: Utilizes a self-correcting workflow to adapt to signal quality.
        \item \textbf{Goal}: Provide scalable, reliable HRV analysis without requiring expert manual intervention.
    \end{itemize}
\end{frame}

% Slide 3: The Signal-to-Insight Gap
\begin{frame}{The Signal-to-Insight Gap}
    \begin{itemize}
        \item \textbf{The Pipeline}: ECG $\rightarrow$ R-peaks detection $\rightarrow$ RR Intervals $\rightarrow$ HRV Metrics.
        \item \textbf{The Challenges}: Noise, baseline drift, motion artifacts, and EMG interference often cause detection failure.
        \item \textbf{The Risk}: Incorrect peak detection leads to invalid RR intervals and distorted HRV results.
        \item \textbf{Interpretation}: Standard metrics (RMSSD, SDNN) are difficult for non-experts to understand without context.
    \end{itemize}
\end{frame}

% Slide 4: Agentic Workflow & Architecture
\begin{frame}{Agentic Workflow \& Architecture}
    \begin{columns}
        \begin{column}{0.5\textwidth}
            \begin{block}{Sense $\rightarrow$ Decide $\rightarrow$ Act $\rightarrow$ Verify}
                \begin{itemize}
                    \item \textbf{Data Ingestion}: PhysioNet + CSV support.
                    \item \textbf{Signal Processing}: Filtering, peak detection, and RR calculation.
                    \item \textbf{Strategy Control}: Validation-driven feedback loop.
                \end{itemize}
            \end{block}
        \end{column}
        \begin{column}{0.5\textwidth}
            \begin{block}{Hybrid Intelligence}
                \begin{itemize}
                    \item \textbf{Deterministic}: NeuroKit2 \& SciPy for signal processing.
                    \item \textbf{Probabilistic}: DeepSeek V3.2 for metric interpretation.
                    \item \textbf{Safety}: Separation of computation vs. interpretation.
                \end{itemize}
            \end{block}
        \end{column}
    \end{columns}
\end{frame}

% Slide 5: Strategy Pattern
\begin{frame}{Strategy Pattern: Why Multiple Pipelines?}
    \begin{itemize}
        \item \textbf{Adaptability}: A single pipeline fails across different signal conditions.
        \item \textbf{Strategy Library}: 
            \begin{itemize}
                \item \textbf{Strategy A}: Standard filtering for clean clinical-grade ECG.
                \item \textbf{Strategy B}: Aggressive filtering for noisy wearable data.
                \item \textbf{Strategy D}: Localized handling for 50 Hz power-line interference.
            \end{itemize}
        \item \textbf{Selection}: The agent tests strategies and selects the best one based on a validation grade.
    \end{itemize}
\end{frame}

% Slide 6: Decision Loop & Validation
\begin{frame}{Decision Loop \& Validation Logic}
    \begin{enumerate}
        \item \textbf{Sensing}: Checks for baseline wander, high-frequency noise, and sampling consistency.
        \item \textbf{Decision}: Try Strategy A $\rightarrow$ Detect Peaks $\rightarrow$ Validate results.
        \item \textbf{Backtracking}: If Grade is "C" or "Reject", switch to Strategy B or D.
        \item \textbf{Criteria}: Physiological HR range, RR consistency, and outlier artifact rejection.
        \item \textbf{Output}: Only accept results that pass physiological plausibility.
    \end{enumerate}
\end{frame}

% Slide 7: Challenges & Results
\begin{frame}{Challenges \& Results}
    \begin{itemize}
        \item \textbf{NumPy 2.0 Compatibility}: Fixed breaking changes in NeuroKit2 by implementing a compatibility shim for \texttt{trapezoid}.
        \item \textbf{Local Data}: Resolved 50 Hz noise issues via Strategy D and timestamp-based sampling inference.
        \item \textbf{Key Deliverables}:
            \begin{itemize}
                \item Clean ECG plots with labeled R-peaks.
                \item Signal quality grading and HRV summary.
                \item Downloadable PDF reports with LLM-generated interpretations.
            \end{itemize}
    \end{itemize}
\end{frame}

% Slide 8: Future Work
\begin{frame}{Future Work}
    \begin{itemize}
        \item Integration of local LLM models to enhance data privacy.
        \item Real-time edge deployment for continuous health monitoring.
        \item Expansion of the strategy library to support more diverse sensor types.
    \end{itemize}
\end{frame}


\begin{frame}{Team Contribution}
    \centering
    \renewcommand{\arraystretch}{1.5}
    \begin{tabular}{|l|c|l|}
        \hline
        \textbf{Member} & \textbf{Contribution} & \textbf{Responsibilities} \\ \hline
        \textbf{Khan} & 40\% & Code Development, Video Recording \\ \hline
        \textbf{Liu} & 30\% & Documentation, Hosting Meeting \\ \hline
        \textbf{Peng} & 30\% & Video Editing, Slide Creation \\ \hline
    \end{tabular}
\end{frame}

\end{document}