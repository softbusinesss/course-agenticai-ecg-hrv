% !TEX program = xelatex
% !TEX encoding = UTF-8 Unicode
%
% The Nordling Lab Presentation example implemented using Beamer by Jose Chang and Torbj�rn Nordling.
% Copyright 2018 by Nordling Lab
%
% You are hereby granted the permission to freely copy and modify this file as you see fit, as long as
% you acknowledges Nordling Lab.
%
% NOTE You need to use XeLaTeX to compile this file!

% Use these lines for the default 4:3 aspect ratio in the presentation
\documentclass{beamer}
\mode<presentation>{
	\usetheme{NordlingLab}
}

% Use these lines for a 16:9 aspect ratio no scaling version in the presentation
%\documentclass[aspectratio=169]{beamer}
%\mode<presentation>{
%	\usetheme{NordlingLab169}
%}

% Load packages
%\usepackage[utf8]{inputenc} %XeLaTeX expects UTF8 so not needed
\usepackage[english]{babel}
\let\latinencoding\relax
\usepackage{xltxtra} %Needed to use Candara font
\setsansfont{Candara}
%\usepackage{fontspec} 
%\setmainfont{Cambria} %Alternative font if Candara is not available
\usepackage{graphicx} %Needed to insert images
\graphicspath{./Figures/}
\usepackage[yyyymmdd]{datetime} %Needed to get date on ISO format
\renewcommand{\dateseparator}{--}

\title[Nordling Lab template]{Official template of the Nordling Lab}
%\subtitle{Add subtitle here}
\author[T.~Nordling]{Ass.~Prof.~Torbj\"{o}rn~E.~M.~Nordling,~Ph.D.}
\institute[NCKU]{Nordling Lab\\
		Dept. of Mechanical Engineering\\
		National Cheng Kung University\\
		No. 1 University rd., Tainan 70101, Taiwan}
\date{\today}
\subject{Beamer presentation template}

% Delete this, if you do not want the table of contents to pop up at
% the beginning of each subsection:
\AtBeginSubsection[]{
  \begin{frame}<beamer>{Outline} % frame is only shown in beamer mode
    \tableofcontents[currentsection,currentsubsection]
  \end{frame}
}


% If you wish to uncover everything in a step-wise fashion, uncomment
% the following command: 
%\beamerdefaultoverlayspecification{<+->}

%%%%%%%%%%%%%%%%%%%%%%%%%%%%%%%%%%%%%%%%%%%%%%%%%%%%%
\begin{document}

\setbeamertemplate{background}[NLTitle] %Use this background on title page
\setbeamertemplate{footline}[NLTitle] %Use this empty footline on title page
\begin{frame}
	\titlepage
\end{frame}

\setbeamertemplate{background}[NLCC] %Use this background to get CC license logo
\setbeamertemplate{footline}[NLCC] %Use this footline to get CC license URL
\begin{frame}<beamer>{Outline} % frame is only shown in beamer mode
	\tableofcontents% You might wish to add the option [pausesections]
\end{frame}

% The reminder of this file is with the exception of the slide containing an image covering all of it 
% and the Nordling Lab theme settings identical to the Beamer example conference-ornate-20min.en.tex
% copyrighted by Till Tantau <tantau@users.sourceforge.net>
% Till Tantau has granted the permission to freely copy and modify his file.

% Structuring a talk is a difficult task and the following structure
% may not be suitable. Here are some rules that apply for this
% solution: 

% - Exactly two or three sections (other than the summary).
% - At *most* three subsections per section.
% - Talk about 30s to 2min per frame. So there should be between about
%   15 and 30 frames, all told.

% - A conference audience is likely to know very little of what you
%   are going to talk about. So *simplify*!
% - In a 20min talk, getting the main ideas across is hard
%   enough. Leave out details, even if it means being less precise than
%   you think necessary.
% - If you omit details that are vital to the proof/implementation,
%   just say so once. Everybody will be happy with that.

\section{Motivation}

\subsection{The Basic Problem That We Studied}

\setbeamertemplate{background}[NL] %Use this on slides containing material that is not under a creative commons license
\setbeamertemplate{footline}[NL][Nordling Lab\\ \url{www.nordlinglab.org} (Last visited 2018-03-01)]%Use this on slides containing material that is not under a creative commons license

\begin{frame}{Make Titles Informative. Be concise.}{Subtitles are optional.}
  % - A title should summarize the slide in an understandable fashion
  %   for anyone how does not follow everything on the slide itself.

  \begin{itemize}
  \item
    Use \texttt{itemize} a lot.
  \item
    Use very short sentences or short phrases.
  \end{itemize}
\end{frame}

% A slide containing an image covering all of it.
\begin{frame}[plain] %Removes any footline
    \begin{centering}%
      \pgfimage[height=0.6\paperheight]{Figures/NordlingLablogo}%
      \par%Trick to get the image centered
    \end{centering}%
\end{frame}
\setbeamertemplate{background}[NLCC] %Use this background to get CC license logo
\setbeamertemplate{footline}[NLCC] %Use this footline to get CC license URL

\begin{frame}[t]
\frametitle{Make Titles Informative.}
  You can create overlays\dots
  \begin{itemize}
  \item using the \texttt{pause} command:
    \begin{itemize}
    \item
      First item.
      \pause
    \item    
      Second item.
    \end{itemize}
  \item
    using overlay specifications:
    \begin{itemize}
    \item<3->
      First item.
    \item<4->
      Second item.
    \end{itemize}
  \item
    using the general \texttt{uncover} command:
    \begin{itemize}
      \uncover<5->{\item
        First item.}
      \uncover<6->{\item
        Second item.}
    \end{itemize}
  \end{itemize}
\end{frame}


\subsection{Previous Work}

\begin{frame}{Make Titles Informative.}
\end{frame}

\begin{frame}{Make Titles Informative.}
\end{frame}



\section{Our Results/Contribution}

\subsection{Main Results}

\begin{frame}{Make Titles Informative.}
\end{frame}

\begin{frame}{Make Titles Informative.}
\end{frame}

\begin{frame}{Make Titles Informative.}
\end{frame}


\subsection{Basic Ideas for Proofs/Implementation}

\begin{frame}{Make Titles Informative.}
\end{frame}

\begin{frame}{Make Titles Informative.}
\end{frame}

\begin{frame}{Make Titles Informative.}
\end{frame}



\section*{Summary}

\begin{frame}{Summary}

  % Keep the summary *very short*.
  \begin{itemize}
  \item
    The \alert{first main message} of your talk in one or two lines.
  \item
    The \alert{second main message} of your talk in one or two lines.
  \item
    Perhaps a \alert{third message}, but not more than that.
  \end{itemize}
  
  % The following outlook is optional.
  \vskip0pt plus.5fill
  \begin{itemize}
  \item
    Outlook
    \begin{itemize}
    \item
      Something you haven't solved.
    \item
      Something else you haven't solved.
    \end{itemize}
  \end{itemize}
\end{frame}



% All of the following is optional and typically not needed. 
\appendix
\section<presentation>*{\appendixname}
\subsection<presentation>*{For Further Reading}

\begin{frame}[allowframebreaks]
  \frametitle<presentation>{For Further Reading}
    
  \begin{thebibliography}{10}
    
  \beamertemplatebookbibitems
  % Start with overview books.

  \bibitem{Author1990}
    A.~Author.
    \newblock {\em Handbook of Everything}.
    \newblock Some Press, 1990.
 
    
  \beamertemplatearticlebibitems
  % Followed by interesting articles. Keep the list short. 

  \bibitem{Someone2000}
    S.~Someone.
    \newblock On this and that.
    \newblock {\em Journal of This and That}, 2(1):50--100,
    2000.
  \end{thebibliography}
\end{frame}

\end{document}
%%%%%%%%%%%%%%%%%%%%%%%%%%%%%%%%%%%%%%%%%%%%%%%%%%%%


