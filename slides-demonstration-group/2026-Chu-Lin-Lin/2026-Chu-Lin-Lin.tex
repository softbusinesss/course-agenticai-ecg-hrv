% !TEX program = xelatex
% !TEX encoding = UTF-8 Unicode
%
% SPDX-License-Identifier: CC-BY-4.0
% Authors: Chu, Lin, Lin

\documentclass[aspectratio=169]{beamer}
\mode<presentation>{
    \usetheme{NordlingLab169}
}

% 載入必要套件
\usepackage[english]{babel}
\usepackage{xeCJK} 
\usepackage{xltxtra}
\usepackage{graphicx}
\usepackage{booktabs}
\usepackage[yyyymmdd]{datetime}
\renewcommand{\dateseparator}{--}

% 字體設定:XeLaTeX 必備
\setsansfont{Arial}
\setCJKmainfont{Microsoft JhengHei} 

% 簡報元數據
\title[Agentic AI for HRV]{Agentic ECG HRV Baseline Evaluation System}
\subtitle{Agentic AI Course - Final Presentation}
% 請記得在此處填入組員的真實名字
\author[Chu et al.]{Chu, Yen-Chieh \and Lin, Chih-Yi \and Lin, Wen-Hsin}
\institute[NCKU]{National Cheng Kung University}
\date{January 19, 2026}

\begin{document}

% --- 1. Title Slide (1 Slide) ---
\setbeamertemplate{background}[NLTitle]
\setbeamertemplate{footline}[NLTitle]
\begin{frame}
    \titlepage
\end{frame}

% --- 2. Outline ---
\setbeamertemplate{background}[NLCC]
\setbeamertemplate{footline}[NLCC]
\begin{frame}{Outline}
    \tableofcontents
\end{frame}

% --- 3. Problem & Motivation (2 Slides) ---
\section{Problem \& Motivation}
\begin{frame}{The Personalization Gap in Healthcare}
    \begin{itemize}
        \item \textbf{Fixed Thresholds}: Universal clinical standards fail to account for individual baseline variations.
        \item \textbf{Activity State Bias}: Physiological signals shift significantly between \textbf{Rest} and \textbf{Active} states.
        \item \textbf{Manual Review Burden}: Technical staff must manually verify massive amounts of discarded ECG data.
    \end{itemize}
\end{frame}

\begin{frame}{Our Motivation: Agentic Solution}
    \begin{itemize}
        \item \textbf{Adaptive Context}: Utilizing Agents to handle multiple activity contexts (Rest / Active).
        \item \textbf{Scalability}: Reducing manual verification through autonomous quality validation loops.
    \end{itemize}
\end{frame}

% --- 4. System Architecture (2 Slides) ---
\section{System Architecture}
\begin{frame}{HRV Analysis Agent Workflow}
    % [T] 確保兩欄都從頂部開始對齊
    \begin{columns}[T]
        % 左側文字欄位
        \begin{column}{0.55\textwidth}
            % 增加一個固定的頂部間距,確保文字不會因為右圖變大而上下晃動
            \vspace{0.2cm} 
            \begin{itemize}
                \item \textbf{Orchestrator}: Coordinates data ingestion and schedules analysis tasks.
                \item \textbf{Adaptive Analyzer}: Selects optimal filtering and R-Peak Detector based on real-time signal quality.
                \item \textbf{Quality Validator}: Verifies HRV metrics against personalized distributions.
            \end{itemize}
        \end{column}
        
        % 右側圖片欄位
        \begin{column}{0.4\textwidth}
            \centering
        	  \hspace*{-1cm}\includegraphics[width=1.4\textwidth, height=0.85\textheight, keepaspectratio]{Figures/2026-Chu-Lin-Lin-system-design.jpg}
        \end{column}
    \end{columns}
\end{frame}

\begin{frame}{Personalized Profile Store}
    \begin{columns}[T]
        \begin{column}{0.95\textwidth} % 使用大寬度確保內容居中
            \vspace{0.5cm}
            \setlength{\leftskip}{0.8cm} % 向右移避開邊界藍線
            
            % 灰色區塊:Personalized Baselines
            \begin{block}{Personalized Baselines (\texttt{baselines.json})}
                Maintains individualized physiological baselines to enable context-aware validation across different activity states.
            \end{block}
            
            \vspace{1em}
            
            % 清單說明
            \begin{itemize}
                \item Stores statistically estimated reference ranges for HRV metrics such as RMSSD, SDNN, and mean heart rate.
                \item Enables the agent to "learn" a user's unique heart rate signature over time.
            \end{itemize}
        \end{column}
    \end{columns}
\end{frame}

% --- 5. Demo & Results (2 Slides) ---
\section{Demo \& Results}

\begin{frame}{Demo \& Results}
    \begin{figure}
        \centering
        % 使用雙引號包裹含空格的路徑,並確保寬度與高度比例適中
        \includegraphics[width=0.85\textwidth,height=0.7\textheight,keepaspectratio]{"Figures/2026-Chu-Lin-Lin-agentic_solution _result_3.jpg"}
    \end{figure}
\end{frame}

% --- 6. Challenges & Lessons Learned (1 Slide) ---
\section{Challenges \& Lessons Learned}
\begin{frame}{Challenges \& Lessons Learned}
    \begin{itemize}
        \item \textbf{Challenge}: Managing the trade-off between Agent reasoning depth and real-time processing speed.
        \item \textbf{Lesson}: Importance of standardized data preprocessing for ECG signals (HRV analysis from ECG).
        \item \textbf{Observation}: Agentic AI significantly reduces "Activity State Bias" compared to static systems.
    \end{itemize}
\end{frame}

% --- 7. Conclusion & Q&A (1 Slide) ---
\section{Conclusion}
\begin{frame}{Summary}
    \begin{itemize}
        \item \textbf{Innovation}: Our system learns individual norms to bridge the monitoring gap.
        \item \textbf{Efficiency}: Adaptive loops reduce the need for manual signal quality verification.
    \end{itemize}
    \vspace{1.5em}
    \centering
    \Large{\textbf{Thank you! Questions?}}
\end{frame}

% --- 8. Contribution List (1 Slide) ---
\section{Team Contribution}
\begin{frame}{Team Contribution}
    \begin{itemize}
        \setlength{\leftskip}{0.5cm} % 稍微向右移避開邊界藍線
        \item \textbf{Chu, Yen-Chieh}: \\ 
        Data-group, Chatbase, Slide, Presentation
        
        \vspace{0.5em}
        
        \item \textbf{Lin, Chih-Yi}: \\ 
        Project-code-group, Tests-group, Agentic approach, Slide, Presentation
        
        \vspace{0.5em}
        
        \item \textbf{Lin, Wen-Hsin}: \\ 
        System-design-group, Chatbase, Presentation
    \end{itemize}
\end{frame}


\end{document}